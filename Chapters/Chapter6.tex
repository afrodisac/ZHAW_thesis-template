% Indicate the main file. Must go at the beginning of the file.
% !TEX root = ../main.tex

%----------------------------------------------------------------------------------------
% CHAPTER 6
%----------------------------------------------------------------------------------------
\chapter{Conclusion} % Main chapter title
\label{Chapter6} % For referencing the chapter elsewhere, use \ref{Chapter1}

\section{Conclusion}
To conclude, In relation to Objective (i), the vast majority of models trained managed to out performed the heuristic as well as the benchmark models, however, most of them predicted no change in water depth. Therefore, in relation to Objective (ii), single time step predictions did not perform as well as anticipated on the unseen rainfall event. After careful consideration of shortcomings and pitfalls, some further evaluation was done on a much simpler model (7000 parameters) and showed much more promising results in relation to Objective (i) and (ii). In relation to Objective (iii), emergent behavior was, in general, not observed, even after further evaluation. However, there were some glimmers of hope where models were able to predict non-linear increase and decrease to water depth based on rainfall, although strange behavior was also noted, where the model seemed to predict the exact shape of the DEM itself. Further evaluation is required to understand this strange behavior but we are optimistic that there is potential for the models to accurately predict water depth recurrently, even if it is not explicitly trained to do so!

\section{Outlook and future work}
Although the project was not a resounding success, some positives were seen in some of the results. The fact that some of the models were able to recurrently predict water depths in a non-linear fashion shows some promise for future evaluations. There are many factors which could have been improved in this thesis and there are many more deep learning techniques to try. Including using a training regime that was used by \citeauthor{growing_nca} where back propagation is done through time. Computational limitations for training was a severe problem. Being forced  to pad small subsections of the training data may have resulted in a loss of critical information. Also the evaluation itself  was flawed and ultimately drove the models to predict no change in water depth. This was shown to be true when positive outcomes resulted from ignoring many of these metrics in a small re-evaluation (see Section \ref{Extra}) was done.

In future work, the use of High Performance Computing Cluster may be necessary. A lot more data could be simulated and run without worrying about padding within the catchment area. More work needs to be done to find a generalizable normalization technique as well as creating a more balanced dataset. Further works also needs to be done on finding a suitable  loss function that drives the model to predict non zero values, which may have been one of the leading cause of error in this thesis. Luckily these are relatively easily remedied. The way the results were obtained and shown, allows for an easy reroute in evaluation.
 