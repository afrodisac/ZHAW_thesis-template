% Indicate the main file. Must go at the beginning of the file.
% !TEX root = ../main.tex

%----------------------------------------------------------------------------------------
% CHAPTER 6
%----------------------------------------------------------------------------------------
\chapter{Conclusion} % Main chapter title
\label{Chapter6} % For referencing the chapter elsewhere, use \ref{Chapter1}

\section{Conclusion}
To conclude, the experiments unfortunately showed extremely poor results. Although the vast majority of models trained out performed the heuristic, they all predicted no change in water depth. So for single time step predictions, where the model has access to the water depth one time step in the past, it performed relatively well. However, this has no practical value as in the real world, all you have access to is the DEM and predicted rainfall event. All models failed to learn properly despite the huge effort of searching hyperparameters.

\section{Outlook and future work}
Although the project was not a success, there were some glimmers of hope in some of the results. The fact that some of the models were able to residually predict water depths in a non-linear fashion shows some promise for future evaluations. There are many factors which could have been improved in this thesis. There are many more deep learning techniques to try. Including using a training regime that was used  by \citeauthor{growing_nca} where back propagation is done through time. Computational limitations for training was a severe problem. Being forced  to pad small subsections of the training data may have resulted in a loss of critical information. \\

In future work, the use of High Performance Computing Cluster may be necessary. A lot more data could be simulated and run without worrying about padding. More work needs to be done to find a generalizable normalization technique as well as creating a less un-balanced dataset. 
 