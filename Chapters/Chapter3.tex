% Indicate the main file. Must go at the beginning of the file.
% !TEX root = ../main.tex

%----------------------------------------------------------------------------------------
% CHAPTER 3
%----------------------------------------------------------------------------------------
\chapter{Flood Modeling} % Main chapter title
\label{Chapter3} % For referencing the chapter elsewhere, use \ref{Chapter1}
\section{Classical Approaches}

There are many different approaches to tackle flood modeling. They include Rapid flood spreading, 1D surface, 1D sewer, 1D-1D coupled, 2D surface and 1D-2D coupled \cite{bulti2020review}. The objectives of this thesis are to model the dynamics and inundation of a flood plain in 2D. Therefore only 2D modeling techniques will be mentioned.

\subsection*{Rapid Flood Spreading} 

Rapid Flood Spreading (RFS)\footnote{The RFS method description has been summarized from \citeauthor{liu2010new} \cite{liu2010new}}, as the name implies, is a simple model to rapidly predict inundation of the flood plain. It works by equalizing the water levels of cells within a local neighbourhood. There are two stages to this process: The pre-calculation and inundation routine. In the pre-calculation routine, low points are identified in the DEM and expanded outwards. In the inundation routine, water is distributed from the inflow source to neighbouring cells using a lowest neighbour principle.

	
\subsection*{Two-Dimensional Surface}
A 2D flood model represents the river network in a mesh, which provides information about the river topography and allows to define very precisely the topography of the floodplain. The hydraulics are solved using two-dimensional shallow  water equations: The continuity equation (Eq. \ref{eq:3.1}), and two-dimensional equations of motion (Eq. \ref{eq:3.2} and Eq. \ref{eq:3.3}) \cite{o1993two}.

\begin{equation}
	\label{eq:3.1}
	\frac{\partial h}{\partial t} + \frac{\partial h V_{x}}{\partial x} + \frac{\partial h V_{y}}{\partial y} = 0
\end{equation}

\begin{equation}
	\label{eq:3.2}
	S_{fx} = S_{ox} - \frac{\partial h}{\partial x} - \frac{V_{x}}{g}\frac{\partial V_{x}}{\partial x} - \frac{V_{y}}{g}\frac{\partial V_{x}}{\partial y} - \frac{1}{g}\frac{\partial V_{x}}{\partial t} 
\end{equation}

\begin{equation}
	\label{eq:3.3}
	S_{fy} = S_{oy} - \frac{\partial h}{\partial y} - \frac{V_{y}}{g}\frac{\partial V_{y}}{\partial y} - \frac{V_{x}}{g}\frac{\partial V_{y}}{\partial x} - \frac{1}{g}\frac{\partial V_{x}}{\partial t} 
\end{equation}
% change the rest of the math 'where' to the following format:
Where, $h$ is flow depth; $g$ is gravitational acceleration; $V_{x}$ and $V_{y}$ are the depth-averaged  velocity components along the $x$ and $y$ coordinates; The friction slope components $S_{fx}$ and $S_{fy}$ are a function of the bed slope $S_{ox}$ and $S_{oy}$, pressure gradient, convective and local acceleration terms. However, the diffusive wave approximation to this equation is defined by neglecting the latter three terms. \\
2D surface models have the advantage of being able to model dynamics of the flood plain, duration and are very accurate.
 
\subsection*{The Problem With Classical Approaches}
RFS is an extremely fast modeling scheme but comes with some major drawbacks. The can only show the final state of inundation and does not describe information about the flood. 2D surface models require fine resolution data and extremely computationally demanding. They can take hours to run and are only able to model small areas. Both of these methods have rather large drawbacks. Especially when unpredictable rainfall events occur and rapid modeling of a flood plain is required to prevent catastrophe. 

\section{The CA Approach}
\label(WCA2D)
The advantage of employing CA for flood modeling is its simplicity \cite{Wolfram2002} and computational speed. It has the advantage of utilizing local functions for computation. Older 2D CA models, although much faster than solving complete shallow water equations still had to solve complicated, computationally demanding equations like Manning's equation (Eq. \ref{eq:manning}).

\subsection*{The Weighted Cellular Automata}
The following is an overview of the paper titled \citetitle{guidolin2016weighted}  by \citeauthor{guidolin2016weighted} \footnote{For a detailed description on the mathematics utilized for this model I refer readers to the original paper \cite{guidolin2016weighted}}. This work is an improvement on the previous CA2D model (A short literature review can be found in Appendix \ref{AppendixA}). This thesis utilized data simulated by the weighted Cellular Automata 2D (WCA2D) for training and validation of deep learning models.\\

\citeauthor{Ghimire}'s paper improved on previous CADDIES-2D model in that instead of directly solving Manning's equation for the computation of interfacial discharges between cells (Eq. \ref{eq:discharge}), a ranking system is used and equation \ref{eq:6} is used. This approach still has issues in that the ranking system equation was still solved for every time step, for each direction to limit the flow velocity. And if the computed velocity was too high, it re-calculated it. This resulted in increased computational time.
\begin{equation}
	\label{eq:manning}
	V = \frac{1}{n} {R}^{2/3}S^{1/2}
\end{equation}
Where $V$ is the cross-sectional average velocity (m/s); $R$ is the hydraulic radius (m); $S$ is hydraulic gradient. To calculate discharge (or volumetric flow rate), $Q = AV$ is used to rewrite Manning's equation to,
\begin{equation}
	\label{eq:discharge}
	Q = A \frac{1}{n} {R}^{2/3}S^{1/2}
\end{equation}

WCA2D is based on the model by \citeauthor{Ghimire} but substitutes the ranking system with a weighted based system that simplifies the transition rules. This transition rule has four steps:
\begin{enumerate}
	\item Identify the downstream neighbour cells
	\item Compute the specific weight of each downstream cell based on the available storage volume
	\item Compute the total amount of volume that will leave the central cell
	\item for each downstream cell, set the eventual intercellular-volume which depends on the previously computed weight and total amount of volume transferred.
\end{enumerate}

Overall the WCA2D model improves performance on the CA2D model but sacrifices some accuracy.
